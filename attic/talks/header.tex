\setbeamercovered{transparent} % was noch kommt, wird schon transparent angezeigt

\usepackage{amssymb,amsmath,amsfonts} %Beamerklasse, Mathesymbole
\usepackage{dsfont}
\usepackage[english]{babel}
\usepackage[utf8]{inputenc} %Codierung für Linux
%\usepackage[ansinew]{inputenc} %Codierung für Windows
\usepackage{bm} % bold symbols in math
\usepackage{appendixnumberbeamer}
%Figure: löschen
\usetheme{simple}
%\setbeamertemplate{footline}[frame number]
\setbeamertemplate{footline}{%
\ifnum \insertpagenumber=1
      \leavevmode%
      \hbox{%
      \begin{beamercolorbox}[wd=\paperwidth,ht=2.25ex,dp=1ex,center]{}%
        % empty environment to raise height
      \end{beamercolorbox}}%
      \vskip0pt%
    \else
      \leavevmode%
      \hbox{%
      \begin{beamercolorbox}[wd=\paperwidth,ht=2.25ex,dp=1ex,right]{title in head/foot}%
        \hfill%
         \usebeamercolor[fg]{page number in head/foot}%
         \usebeamerfont{title in head/foot}%
         { \insertframenumber\,/\,\inserttotalframenumber\kern1em}
          \includegraphics[height=1cm]{tidyfun.png}
      \end{beamercolorbox}}%
      \vskip0pt%
    \fi


% \ifnum \insertpagenumber=1
%   \leavevmode%
%       \hbox{%
%       \begin{beamercolorbox}[wd=\paperwidth,ht=2.25ex,dp=1ex,center]{}%
%         % empty environment to raise height
%       \end{beamercolorbox}}%
%       \vskip0pt%
% \else
%    \leavevmode%
%       \hbox{%
%       \begin{beamercolorbox}[wd=\paperwidth,ht=2.25ex,dp=1ex,center]{
%         \flushleft\includegraphics[height=0.3cm]{tidyfun.png}%
%         \hfill%
%         \usebeamercolor[fg]{page number in head/foot}%
%         \usebeamerfont{page number in head/foot}%
%         {\scriptsize \insertframenumber\,/\,\inserttotalframenumber\kern1em}}%
%        \end{beamercolorbox}}%
%       \vskip0pt%
% \fi
 }

\setbeamertemplate{caption}{\insertcaption}
\setbeamertemplate{caption label separator}{}
\setbeamertemplate{blocks}[rounded][shadow=true]
\beamertemplatenavigationsymbolsempty

\definecolor{lmuGrau}{RGB}{160,160,160}
\definecolor{myGrey}{RGB}{240,240,240}
\definecolor{lmugreen}{RGB}{21,64,157} %blau
\definecolor{lmuGruen}{RGB}{21,64,157} %blau
\definecolor{mycol}{RGB}{255,51,0} % rot
\definecolor{mygreen}{RGB}{30,180,150} % gr?n bis t?rkis

\newcommand{\green}[1]{\color{lmuGruen}#1}
\newcommand{\grey}[1]{\color{lmuGrau}#1}
\newcommand{\mygreen}[1]{\color{mygreen}#1}
\newcommand{\blue}[1]{\color{blue}#1}
\newcommand{\red}[1]{\color{mycol}#1}
\newcommand{\spacing}{\setlength{\itemsep}{10pt}}
\newcommand{\fb}{\framebreak}

% % KNITR DEFS
% %-------------------------------------------------------------------------------
% \usepackage{color}
% \usepackage{fancyvrb}
% \newcommand{\VerbBar}{|}
% \newcommand{\VERB}{\Verb[commandchars=\\\{\}]}
% \DefineVerbatimEnvironment{Highlighting}{Verbatim}{commandchars=\\\{\}}
% % Add ',fontsize=\small' for more characters per line
% \usepackage{framed}
% \definecolor{shadecolor}{RGB}{248,248,248}
% \newenvironment{Shaded}{\begin{snugshade}}{\end{snugshade}}
% \newcommand{\KeywordTok}[1]{\textcolor[rgb]{0.13,0.29,0.53}{\textbf{#1}}}
% \newcommand{\DataTypeTok}[1]{\textcolor[rgb]{0.13,0.29,0.53}{#1}}
% \newcommand{\DecValTok}[1]{\textcolor[rgb]{0.00,0.00,0.81}{#1}}
% \newcommand{\BaseNTok}[1]{\textcolor[rgb]{0.00,0.00,0.81}{#1}}
% \newcommand{\FloatTok}[1]{\textcolor[rgb]{0.00,0.00,0.81}{#1}}
% \newcommand{\ConstantTok}[1]{\textcolor[rgb]{0.00,0.00,0.00}{#1}}
% \newcommand{\CharTok}[1]{\textcolor[rgb]{0.31,0.60,0.02}{#1}}
% \newcommand{\SpecialCharTok}[1]{\textcolor[rgb]{0.00,0.00,0.00}{#1}}
% \newcommand{\StringTok}[1]{\textcolor[rgb]{0.31,0.60,0.02}{#1}}
% \newcommand{\VerbatimStringTok}[1]{\textcolor[rgb]{0.31,0.60,0.02}{#1}}
% \newcommand{\SpecialStringTok}[1]{\textcolor[rgb]{0.31,0.60,0.02}{#1}}
% \newcommand{\ImportTok}[1]{#1}
% \newcommand{\CommentTok}[1]{\textcolor[rgb]{0.56,0.35,0.01}{\textit{#1}}}
% \newcommand{\DocumentationTok}[1]{\textcolor[rgb]{0.56,0.35,0.01}{\textbf{\textit{#1}}}}
% \newcommand{\AnnotationTok}[1]{\textcolor[rgb]{0.56,0.35,0.01}{\textbf{\textit{#1}}}}
% \newcommand{\CommentVarTok}[1]{\textcolor[rgb]{0.56,0.35,0.01}{\textbf{\textit{#1}}}}
% \newcommand{\OtherTok}[1]{\textcolor[rgb]{0.56,0.35,0.01}{#1}}
% \newcommand{\FunctionTok}[1]{\textcolor[rgb]{0.00,0.00,0.00}{#1}}
% \newcommand{\VariableTok}[1]{\textcolor[rgb]{0.00,0.00,0.00}{#1}}
% \newcommand{\ControlFlowTok}[1]{\textcolor[rgb]{0.13,0.29,0.53}{\textbf{#1}}}
% \newcommand{\OperatorTok}[1]{\textcolor[rgb]{0.81,0.36,0.00}{\textbf{#1}}}
% \newcommand{\BuiltInTok}[1]{#1}
% \newcommand{\ExtensionTok}[1]{#1}
% \newcommand{\PreprocessorTok}[1]{\textcolor[rgb]{0.56,0.35,0.01}{\textit{#1}}}
% \newcommand{\AttributeTok}[1]{\textcolor[rgb]{0.77,0.63,0.00}{#1}}
% \newcommand{\RegionMarkerTok}[1]{#1}
% \newcommand{\InformationTok}[1]{\textcolor[rgb]{0.56,0.35,0.01}{\textbf{\textit{#1}}}}
% \newcommand{\WarningTok}[1]{\textcolor[rgb]{0.56,0.35,0.01}{\textbf{\textit{#1}}}}
% \newcommand{\AlertTok}[1]{\textcolor[rgb]{0.94,0.16,0.16}{#1}}
% \newcommand{\ErrorTok}[1]{\textcolor[rgb]{0.64,0.00,0.00}{\textbf{#1}}}
% \newcommand{\NormalTok}[1]{#1}
% \usepackage{graphicx,grffile}
% \makeatletter
% \def\maxwidth{\ifdim\Gin@nat@width>\linewidth\linewidth\else\Gin@nat@width\fi}
% \def\maxheight{\ifdim\Gin@nat@height>\textheight0.8\textheight\else\Gin@nat@height\fi}
% \makeatother
% % Scale images if necessary, so that they will not overflow the page
% % margins by default, and it is still possible to overwrite the defaults
% % using explicit options in \includegraphics[width, height, ...]{}
% \setkeys{Gin}{width=\maxwidth,height=\maxheight,keepaspectratio}
%
% %-------------------------------------------------------------------------------

\def\bcs{\begin{columns}}
\def\ecs{\end{columns}}
\def\bc{\begin{column}{0.5\textwidth}}
\def\ec{\end{column}}

\titlegraphic{
%\begin{columns}
% \begin{column}[b]{0.45\textwidth}
%     \begin{figure}%
%     \includegraphics[width=0.5\columnwidth]{pics/lmu_logo2}%
%     \end{figure}
% \end{column}
% \begin{column}[b]{0.45\textwidth}
%     \begin{figure}%
%     \includegraphics[width=0.5\columnwidth]{pics/logo_emmy_noether_203}%
%     \end{figure}
% \end{column}
% \end{columns}
 \flushright
 \includegraphics[width=.2\textwidth,height=.2\textheight]{tidyfun.png}
 \vskip 5em
}

\AtBeginSubsection[] {
    \begin{frame}[plain]{Outline}
    \tableofcontents[currentsection,currentsubsection]
    \end{frame}
    \addtocounter{framenumber}{-1}
}

% https://stackoverflow.com/questions/38323331/code-chunk-font-size-in-beamer-with-knitr-and-latex
%% change fontsize of R code
\let\oldShaded\Shaded
\let\endoldShaded\endShaded
\renewenvironment{Shaded}{\scriptsize\oldShaded}{\endoldShaded}

%% change fontsize of output
\let\oldverbatim\verbatim
\let\endoldverbatim\endverbatim
\renewenvironment{verbatim}{\scriptsize\oldverbatim}{\endoldverbatim}

% https://stackoverflow.com/questions/34866163/decreasing-space-between-commands-and-output-in-knitr-chunks?noredirect=1&lq=1
\usepackage{etoolbox}
\makeatletter
\preto{\@verbatim}{\topsep=0pt \partopsep=0pt }
\makeatother
